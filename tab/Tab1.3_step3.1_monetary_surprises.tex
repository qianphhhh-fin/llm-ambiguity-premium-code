
\begin{table}[h!]
\centering
\caption{Narrative Ambiguity vs. Monetary Policy Surprises}
\label{tab:mp_shock_horse_race}
\begin{threeparttable}
\begin{tabular}{lccc}
\toprule
 & \multicolumn{3}{c}{Dependent Variable: $\Delta \text{VIX}_{t}$} \\
\cmidrule(lr){2-4}
Variable & (1) & (2) & (3) \\
\midrule
\multicolumn{4}{l}{\textbf{Panel A: Regression Analysis}} \\
    \hspace{1em}Narrative Shock ($\Delta D_t$) & 0.142 & & 0.120 \\
         & (0.59) & & (0.49) \\
    \hspace{1em}MP Shock Magnitude ($|MPS_t|$) & & -0.289 & -0.280 \\
         & & (-1.40) & (-1.28) \\
\midrule
Constant & Yes & Yes & Yes \\
Observations & 210 & 210 & 210 \\
Adj. $R^2$ & -0.002 & 0.008 & 0.005 \\
\midrule
\multicolumn{4}{l}{\textbf{Panel B: Orthogonality Check}} \\
    \hspace{1em}Corr($\Delta D_t$, $|MPS_t|$) & \multicolumn{3}{c}{-0.078} \\
\bottomrule
\end{tabular}
\begin{tablenotes}[para,flushleft]
  \item Note: This table examines whether narrative ambiguity is simply a proxy for the magnitude of monetary policy surprises. MP Surprises are from Acosta et al. (2025), using the high-frequency shock around the FOMC Statement release. We use the absolute value of the shock ($|MPS_t|$) to capture the magnitude of the surprise regardless of direction. Newey-West standard errors (1 lag) are used.
\end{tablenotes}
\end{threeparttable}
\end{table}
